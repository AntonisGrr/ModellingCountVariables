%============================================================%

\documentclass[MASTER.tex]{subfiles}
\begin{document}
	%============================================================%
	\begin{frame}
		\frametitle{Zero Truncated Poisson Distribution}
		\Large
		\textbf{Zero-Truncated Poisson Regression}
		\begin{itemize}
		\item Zero-truncated Modelling is used to model count data for which the value zero cannot occur.
		\end{itemize}
		
		\end{frame}
%==================================================================== %
\begin{frame}
	\frametitle{Examples of Zero-Truncated Model}
	\textbf{Example 1.}
	\begin{itemize}
		\item A study of length of hospital stay, in days, as a function of age, kind of health insurance and whether or not the patient died while in the hospital. 
		\item Length of hospital stay is recorded as a minimum of at least one day.
	\end{itemize}
\end{frame}
%==================================================================== %
\begin{frame}
	\frametitle{Examples of Zero-Truncated Model}
	\textbf{Example 2.}
	\begin{itemize} 
		\item A study of the number of journal articles published by tenured faculty as a function of discipline (fine arts, science, social science, humanities, medical, etc).
		\item To get tenure faculty must publish, therefore, there are no tenured faculty with zero publications.
	\end{itemize}
\end{frame}
%==================================================================== %
\begin{frame}
	\frametitle{Examples of Zero-Truncated Model}
	\textbf{Example 3.}
	\begin{itemize} 
		\item A study by the county traffic court on the number of tickets received by teenagers as predicted by school performance, amount of driver training and gender.
		\item Only individuals who have received at least one citation are in the traffic court files.
	\end{itemize}
\end{frame}
%==================================================================== %
\begin{frame}
	\frametitle{Examples of Zero-Truncated Model}
	\textbf{Example 4.}
\begin{itemize}
	\item 
Consider for example the random variable of the number of items in a shopper's basket at a supermarket checkout line. 
\item Presumably a shopper does not stand in line with nothing to buy (i.e. the minimum purchase is 1 item), so this phenomenon may follow a ZTP distribution
	\end{itemize}
\end{frame}

\end{document}