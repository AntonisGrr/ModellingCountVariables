\documentclass[00-GLMregslides.tex]{subfiles}

\begin{document}

%==========================================%
\begin{frame}[fragile]

\frametitle{Poisson Regression with \texttt{R}}
\Large
\textbf{Over-Dispersion}\\ 
\begin{itemize}
\item Overdispersion is the presence of greater variability
 in a data set than would be expected based on a given simple statistical model.
\item Poisson Distribution:
\[ \mathrm{Var}(X) >  \mathrm{E}(X) \]
\end{itemize}
\end{frame}
%==========================================%

%==========================================%
\begin{frame}[fragile]
	\frametitle{Poisson Regression with \texttt{R}}
	\Large
  \textbf{Zero-Inflation}
	\begin{itemize}
		\item One common cause of over-dispersion is excess zeros, which in turn are generated by an additional data generating 
		process. 
		\item In this situation, zero-inflated model should be considered.
		\item If the data generating process does not allow for any 0s (such as the number of days spent in the hospital), then a zero-truncated model may be more appropriate.
	\end{itemize}
\end{frame}
%==========================================%
\begin{frame}[fragile]

\frametitle{Poisson Regression with \texttt{R}}
\Large
\textbf{Over-Dispersion}
\begin{itemize}

 \item When there seems to be an issue of dispersion, we should first check if our model is appropriately specified, 
 such as omitted variables and functional forms. 
 \item For example, if we omitted the predictor variable prog in the example above, our model would seem to have a 
 problem with over-dispersion. 
 \item In other words, a misspecified model could present a symptom like an over-dispersion problem. 
\end{itemize}
\end{frame}
%==========================================%
\begin{frame}[fragile]
\frametitle{Poisson Regression with \texttt{R}}
	\Large
\begin{itemize}
\item Assuming that the model is correctly specified, the assumption that the conditional variance is equal 
to the conditional mean should be checked. 
\item There are several tests including the likelihood ratio test of over-dispersion parameter alpha by running the 
same model using negative binomial distribution. 
\item The \texttt{R} package \textbf{pscl} (Political Science Computational Laboratory, Stanford University) provides many functions for binomial 
and count data including \texttt{odTest} for testing over-dispersion. 
\end{itemize}
\end{frame}
%==========================================%
\begin{frame}[fragile]
	\frametitle{Poisson Regression with \texttt{R}}
	\Large
	\begin{itemize} 
\item Count data often have an exposure variable, which indicates the number of times the event could have happened. 
\item This variable should be incorporated into a Poisson model with the use of the offset option.
\item The outcome variable in a Poisson regression cannot have negative numbers, and the exposure cannot have 0s.
\end{itemize}
\end{frame}
%==========================================%
\begin{frame}[fragile]
	\frametitle{Poisson Regression with \texttt{R}}
	\Large
	\begin{itemize} 

\item Many different measures of pseudo-R-squared exist. They all attempt to provide information similar to that provided by R-squared in OLS regression, even though none of them can be interpreted exactly as R-squared in OLS regression is interpreted.
 
%For a discussion of various pseudo-R-squares, see Long and Freese (2006) or our FAQ page What are pseudo R-squareds?.
\item Poisson regression is estimated via maximum likelihood estimation. It usually requires a large sample size. 
\end{itemize}
\end{frame}

%================================================================================================%
\end{document}
