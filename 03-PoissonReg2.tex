\documentclass[00-GLMregslides.tex]{subfiles}
\begin{document}
	
	%================================================================================================%
	\begin{frame}[fragile]
		
		\frametitle{Poisson Regression with \texttt{R}}
		\Large
		
		\textbf{Poisson regression}
		\begin{itemize}
			\item At this point, we are ready to perform our Poisson model analysis using the \texttt{glm} function. 
			\item We fit the model and store it in the object \texttt{model1} and get a summary of the model.
		\end{itemize}
	\end{frame}
	
	%================================================================================================%
	\begin{frame}[fragile]
		
		\frametitle{Poisson Regression with \texttt{R}}
		\large
		\begin{framed}
			\begin{verbatim}
			model1 <- glm(num_awards ~ prog + math, 
			family="poisson", data=p)
			
			summary(model1)
			\end{verbatim}
		\end{framed}
	\end{frame}
	%================================================================================================%
	\begin{frame}[fragile]
		\frametitle{Poisson Regression with \texttt{R}}
		\large
		\begin{framed}
			\begin{verbatim}
			
			Call:
			glm(formula = num_awards ~ prog + math, family = "poisson", data = p)
			Deviance Residuals: 
			Min      1Q  Median      3Q     Max  
			-2.204  -0.844  -0.511   0.256   2.680  
			\end{verbatim}
		\end{framed}
	\end{frame}
	
	%================================================================================================%
	\begin{frame}[fragile]
		
		\frametitle{Poisson Regression with \texttt{R}}
		\large
		\begin{framed}
			\begin{verbatim}
			
			Coefficients:
			Estimate Std. Error z value Pr(>|z|)    
			(Intercept)     -5.2471     0.6585   -7.97  1.6e-15 ***
			progAcademic     1.0839     0.3583    3.03   0.0025 ** 
			progVocational   0.3698     0.4411    0.84   0.4018    
			math             0.0702     0.0106    6.62  3.6e-11 ***
			---
			Signif. codes:  0 '***' 0.001 '**' 0.01 '*' 0.05 '.' 0.1 ' ' 1
			
			\end{verbatim}
		\end{framed}
	\end{frame}
	
	%================================================================================================%
	\begin{frame}[fragile]
		
		\frametitle{Poisson Regression with \texttt{R}}
		\large
		\begin{framed}
			\begin{verbatim}
			
			(Dispersion parameter for poisson family taken to be 1)
			
			Null deviance: 287.67  on 199  degrees of freedom
			Residual deviance: 189.45  on 196  degrees of freedom
			AIC: 373.5
			
			Number of Fisher Scoring iterations: 6
			
			\end{verbatim}
		\end{framed}
	\end{frame}
	
	%================================================================================================%
	\begin{frame}[fragile]
		
		\frametitle{Poisson Regression with \texttt{R}}
		\Large 
		\begin{itemize}
			% Cameron and Trivedi (2009) 
			\item It is recommended using robust standard errors for the parameter estimates to control for mild violation of the distribution assumption that the variance equals the mean. 
			\item The \texttt{R} package \textbf{\textit{sandwich}} can be used to obtain the robust standard errors and calculated the p-values accordingly. 
			\item Together with the p-values, we have also calculated the 95\% confidence interval using the parameter estimates and their robust standard errors. 
		\end{itemize}
	\end{frame}
	
	%================================================================================================%
	\begin{frame}[fragile]
		
		\frametitle{Poisson Regression with \texttt{R}}
		\large
		\textbf{sandwich \texttt{R} Package }
		\begin{itemize}
			\item Robust Covariance Matrix Estimators
			
			\item Model-robust standard error estimators for cross-sectional, time series, and longitudinal data.
			
		\end{itemize}
	\end{frame}
	
	%================================================================================================%
	\begin{frame}[fragile]
		
		\frametitle{Poisson Regression with \texttt{R}}
		\large
		\textbf{Robust Standard Errors}
		\begin{framed}
			\begin{verbatim}
			cov.model1 <- vcovHC(model1, type="HC0")
			std.err <- sqrt(diag(cov.model1))
			
			r.est <- cbind(Estimate= coef(model1), 
			"Robust SE" = std.err,
			"Pr(>|z|)" = 2 * pnorm(abs(coef(model1)/std.err), 
			lower.tail=FALSE),
			LL = coef(model1) - 1.96 * std.err,
			UL = coef(model1) + 1.96 * std.err)
			
			\end{verbatim}
		\end{framed}
	\end{frame}
	
	%================================================================================================%
	\begin{frame}[fragile]
		
		\frametitle{Poisson Regression with \texttt{R}}
		\large 
		
			\begin{verbatim}
			r.est
			
			Estimate Robust SE  Pr(>|z|)      LL       UL
			(Intercept)    -5.24712   0.64600 4.567e-16 -6.5133 -3.98097
			progAcademic    1.08386   0.32105 7.355e-04  0.4546  1.71311
			progVocational  0.36981   0.40042 3.557e-01 -0.4150  1.15463
			math            0.07015   0.01044 1.784e-11  0.0497  0.09061
			\end{verbatim}
	
		
	\end{frame}
	
	%================================================================================================%
\end{document}
