%============================================================%
\begin{frame}
\frametitle{Zero Truncated Poisson Distribution}
\Large

R Data Analysis Examples: Zero-Truncated Poisson Regression

Zero-truncated poisson regression is used to model count data for which the value zero cannot occur.

This page uses the following packages. Make sure that you can load them before trying to run the examples on this page. If you do not have a package installed, run: install.packages("packagename"), or if you see the version is out of date, run: update.packages().

require(foreign)
require(ggplot2)
require(VGAM)
require(boot)
Version info: Code for this page was tested in R Under development (unstable) (2012-11-16 r61126)
On: 2012-12-15
With: boot 1.3-7; VGAM 0.9-0; ggplot2 0.9.3; foreign 0.8-51; knitr 0.9

Please Note: The purpose of this page is to show how to use various data analysis commands. It does not cover all aspects of the research process which researchers are expected to do. In particular, it does not cover data cleaning and verification, verification of assumptions, model diagnostics and potential follow-up analyses.
\end{frame}
%============================================================%
\begin{frame}
\frametitle{Zero Truncated Poisson Distribution}
\Large

The zero-truncated Poisson (ZTP) distribution is a certain discrete probability distribution whose support is the set of positive integers. This distribution is also known as the conditional Poisson distribution[1] or the positive Poisson distribution.[2] It is the conditional probability distribution of a Poisson-distributed random variable, given that the value of the random variable is not zero. Thus it is impossible for a ZTP random variable to be zero. Consider for example the random variable of the number of items in a shopper's basket at a supermarket checkout line. Presumably a shopper does not stand in line with nothing to buy (i.e. the minimum purchase is 1 item), so this phenomenon may follow a ZTP distribution.[3]

\end{frame}
%============================================================%
\begin{frame}

Since the ZTP is a truncated distribution with the truncation stipulated as k > 0, one can derive the probability mass 
function g(k;λ) from a standard Poisson distribution f(k;λ) as follows: [4]
g(k;\lambda) = P(X = k \mid k > 0) = 
\frac{f(k;\lambda)}{1-f(0;\lambda)} = 
\frac{\lambda ^ k e^{- \lambda} }{k ! \left ( 1 - e^{- \lambda} \right )} = \frac{\lambda^k}{(e^\lambda-1)k!}

\end{frame}
%============================================================%

The mean is
 \operatorname{E}[X]=\frac{\lambda}{1-e^{-\lambda}}=\frac{\lambda e^\lambda}{e^\lambda-1} 
and the variance is
 \operatorname{Var}[X]=\frac{\lambda}{1-e^{-\lambda}} - \frac{\lambda^2 e^{-\lambda}}{(1-e^{-\lambda})^2} =
\frac{\lambda e^\lambda}{e^\lambda-1}\left[1-\frac{\lambda}{e^\lambda-1}\right]

\end{frame}
%============================================================%
\end{document}


\end{document}
