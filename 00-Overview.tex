\documentclass[MASTER.tex]{subfiles}
\begin{document}
\begin{frame}
	\Large
\begin{itemize}
\item This talks is about regression methods in which the dependent variable takes
nonnegative integer or count values. \item The dependent variable is usually the number of times an event occurs. 
\end{itemize}
\end{frame}
%======================================================== %
\begin{frame}
	\frametitle{Overview}
	\Large Some
examples of event counts are:
\begin{itemize}
\item number of claims per year on a particular car owner’s insurance policy,
\item number of workdays missed due to sickness of a dependent in a one-year period,
\item number of papers published per year by a researcher.
\end{itemize}
\end{frame}
%==================================================== %
\begin{frame}
	\frametitle{Poisson Distribution: Prussian Cavalary}
\Large
\begin{itemize}
\item The classic Poisson example is the data set of von Bortkiewicz (1898), for the chance of a Prussian cavalryman being killed by the kick of a horse. 
\item Ten army corps were observed over 20 years, giving a total of 200 observations of one corps for a one year period. The period or module of observation is thus one year. 


\end{itemize}
\end{frame}
%======================================================== %
\begin{frame}
	\frametitle{Poisson Distribution: Prussian Cavalary}
	\Large
	\begin{itemize}
\item The total deaths from horse kicks were 122, and the average number of deaths per year per corps was thus 122/200 = 0.61. 
%\item This is a rate of less than 1. It is also obvious that it is meaningless to ask how many times per year a cavalryman was not killed by the kick of a horse. 
\item In any given year, we expect to observe, well, not exactly 0.61 deaths in one corps 
%(that is not possible; deaths occur in modules of 1), but sometimes none, sometimes one, occasionally two, perhaps once in a while three, and (we might intuitively expect) very rarely any more. 
\item Here, then, is the classic Poisson situation: a rare event, whose average rate is small, with observations made over many small intervals of time.
\end{itemize}
\end{frame}
%======================================================== %
\begin{frame}
\frametitle{Overview}
\Large
\begin{itemize}
\item  Poisson regression is main technique used to model count variables.
\item  Sometimes conventional Poisson Regression is not an appropriate technique, and alternative or variant techniques are used instead.
\item  Negative Binomial regression is for modelling count variables, usually for over-dispersed count outcome variables.
\end{itemize}
\end{frame}
%=========================================================== %

\begin{frame}[fragile]
\frametitle{Generalized Linear Models}\large
\begin{itemize}
\item In statistics, the problem of modelling count variables is an example of generalized linear modelling.
\item Generalized linear models are fit using the \texttt{glm()} function. 
\item The form of the \texttt{glm} function is
\end{itemize}
{
\normalsize
\begin{framed}
 \begin{verbatim}
glm(formula, family=familytype(link=linkfunction),
  data=dataname)
\end{verbatim}
\end{framed}
}
\end{frame}
%========================================================== %
\begin{frame}[fragile]
	\frametitle{Generalized Linear Models}
	\large
\begin{center}
\begin{tabular}{|c|l|} \hline
	Family &	Default Link Function \\ \hline \hline 
	binomial&	\texttt{(link = "logit")} \\ \hline
	gaussian&	\texttt{(link = "identity")} \\ \hline
	Gamma&	\texttt{(link = "inverse")} \\ \hline
	inverse.gaussian&	\texttt{(link = "$1/mu^2$")} \\ \hline
	\textbf{poisson}	&\texttt{(link = "log")} \\ \hline
	%quasi&	\texttt{(link = "identity", variance = "constant")} \\ \hline
	quasibinomial&	\texttt{(link = "logit")} \\ \hline
	quasipoisson&	\texttt{(link = "log")} \\ \hline
	
\end{tabular}
\end{center}
 
\end{frame}

\begin{frame}
	\large
	\textbf{Texts on GLMs}\\ \bigskip
\begin{itemize}
\item Dobson, A. J. (1990) An Introduction to Generalized Linear Models. (\textit{London: Chapman and Hall}.)
\bigskip
\item Hastie, T. J. and Pregibon, D. (1992) Generalized linear models. Chapter 6 of Statistical Models in S eds J. M. Chambers and T. J. Hastie, Wadsworth \& Brooks/Cole.
\bigskip
\item McCullagh P. and Nelder, J. A. (1989) Generalized Linear Models. (\textit{London: Chapman and Hall.})
\bigskip
\item Venables, W. N. and Ripley, B. D. (2002) Modern Applied Statistics with S. \textit{New York: Springer.}
\end{itemize}

\end{frame}
%========================================================== %
\begin{frame}
\textbf{VGAM: Vector Generalized Linear and Additive Models}
\begin{description}
\item[Author(s):] Thomas W. Yee (t.yee@auckland.ac.nz)
\item[License:] GPL-2
\item[URL:] http://www.stat.auckland.ac.nz/$\sim$ yee/VGAM
\end{description}
Vector generalized linear and additive models, and associated models (Reduced-Rank VGLMs, Quadratic RR-VGLMs, Reduced-Rank VGAMs). \\ \bigskip This package fits many models and distribution by maximum likelihood estimation (MLE) or penalized MLE. Also fits constrained ordination models in ecology.
\end{frame}
%================================================================== %
\end{document}






